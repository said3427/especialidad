\documentclass[11pt]{article}
\usepackage{fancyhdr}
\usepackage{enumerate}
\usepackage{titlesec}
\usepackage{titletoc}
\usepackage[spanish]{babel}
\usepackage{latexsym}
\usepackage[utf8]{inputenc}
\usepackage{amsfonts}
\usepackage{amsmath}
\usepackage{amssymb}
\usepackage{amsthm}
\usepackage[dvips]{graphicx}
\begin{document}
\section*{Ejercicio 6}
Mostrar que la distribución \(N(\mu, \tau)\) pertenece a la familia exponencial. A partir de este resultado,
encontrar un estadistico suficiente para \((\mu, \tau)\).
\subsection*{Respuesta}
Una distribución de probabilidad pertenece a la familia exponencial si su función de densidad de probabilidad se puede escribir como:

\begin{equation}
f_{X}(x ; \theta)=h(x) c(\theta) e^{\sum_{n=1}^{k} \omega_{i}(\theta) t_{i}(x)} 
\end{equation}

Donde, \(h(x) \text{ y }  t(x)\) son funciones reales de x y no depende de \(\theta,\) mientras que \(h(\theta) \text{ y } c(\theta)\) son funciones que dependen de \(\theta\).

La función de densidad de probabilidad para \(N(\mu, \tau)\), donde \(\mu, \tau\) son desconocidos:
$$
f_{X}(x; \mu ,\tau)=\frac{1}{\sqrt{2 \pi \tau}} e^{-\frac{1}{2 \tau}(x-\mu)^{2}} \quad x \in \mathbb{R}
$$

$$
f_{X}(x; \mu ,\tau)=\frac{1}{\sqrt{2 \pi \tau}} e^{-\frac{1}{2 \tau}(x^2-2\mu x +\mu^2)} \quad x \in \mathbb{R}
$$

$$
f_{X}(x; \mu ,\tau)=\frac{1}{\sqrt{2 \pi \tau}} e^{-\frac{\mu^2}{2 \tau}} e^{-\frac{1}{2 \tau}(x^2-2\mu x )} \quad x \in \mathbb{R}
$$

$$
f_{X}(x; \mu ,\tau)=\frac{1}{\sqrt{2 \pi \tau}} e^{-\frac{\mu^2}{2 \tau}} e^{-\frac{x^2}{2 \tau}+\frac{\mu x}{{\tau}} } \quad x \in \mathbb{R}
$$

Se puede reescribir la ecuación 1 de la siguiente forma:
$$c(\theta)=\frac{1}{\sqrt{2 \pi \tau}} e^{-\frac{\mu^2}{2 \tau}} \text{;} h(x) = 1$$

$$w_{1} = -\frac{\mu}{\tau} ;  t_{1}= x$$

$$ w_{2} = -\frac{1}{2 \tau} ;  t_{2}= x^2$$

Se comprueba que la distribución \(N(\mu, \tau)\) pertenece a la familia exponencial.
\subsubsection*{Estadístico suficiente}

Una vez que se mostró que la distribución \(N(\mu, \tau)\) pertenece a la familia exponencial y con el teorema de factorización se obtienen los estadísticos suficientes para \(\theta (\mu, \tau)\) 

$$
T(X)= \begin{pmatrix} \sum_{1}^n X \\\sum_{1}^n X^2 \end{pmatrix}
$$

\end{document}